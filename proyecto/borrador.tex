El modelo de Lotka-Volterra es un sistema de ecuaciones diferenciales de primer orden no lineales que se usa para describir la dinamica de un ecosistema en el que dos especies interactuan (presa y depredador)

El modelo se escribe de la siguiente forma:
dS/dt = S(a-bF)
dF/dt = -F(c-dS) 
a, b, c, d son parametros positivos que representan las interacciones de las especies.
(Foxes and sheeps)

a constante de crecimiento (intrinseca) de S
c tasa de muerte de F
b interaccion entre dos especies desfavorables para la presa (signo -)
d interaccion favorable para el depredador

%-----------------------------------------


Se simula el modelo de Lotka-Volterra, mejor conocido como modelo depredador-presa utilizando un sistema multi-agente.
Se consideran dos estados: lobos y ovejas.

Al inicio tengo N = f + s
N poblacion inicial total
f poblacion lobos (foxes)
s poblacion ovejas (sheeps)

Los lobos comeran a las ovejas que se encuentren en cierto rango

Si comen: aumenta poblacion lobos y disminuye ovejas
Si NO comen: disminuye polacion

Si NO son comidos: aumenta poblacion ovejas

%-----------------------------------------

Para la velocidad, se debe tomar en cuenta la especie
Tomar en cuenta tamaño de ambas especies
Considerar hembras y machos para que se logre reproduccion
Hembra embarazada debe vivir cierto tiempo para que haya crias
Lobos grandes necesitan mas alimento
Considera pasto
Y si las ovejas intentan escapar del lobo??
Lobos no comeran todo el tiempo, comen y luego descanzan
Puedo simplificar para no complicarme con el emparejamiento y suponer que todos se reproducen milagrosamente solos

OBJETIVO:
una vez hecha la simulacion, se desea observar si esta tiene el mismo comportamiento que se observa de las ecuaciones

Resutados numericos mandarlos a un archivo

Antecedentes del tema
Que desarrolle
Experimentacion
Resultados
Conclusiones

En el repo va todo lo que yo produzco

%------------------------------------------

El comportamiento que se desea obtener es el siguiente:

- Si no hay depredador, el crecimiento de las presas se da de manera exponencial
- Si no hay alimento, la muerte del depredador se da de manera exponencial
- Ciclos: si el depredador come mucho, cada vez habra menos alimento, asi que comenzaran a morir. Una vez que disminuya la poblacion de depredadores, la poblacion de presas incrementa ya que no hay quien se las coma.

%-------------------------------------------

14 noviembre
Primero que funcione, luego veo hacerlo efectivo
incluso podria agregarlo a reporte

Pensar un a funcion tipo sierra que me dice las ganas de comer de los lobos (pueden compartir comida!!)
(que dependa del tiempo)
Sino come: incrementa su hambre 
Si come: disminuye
Si llega a un limite su hambre: darle cuello
Incluir si se puede:
- resistencia del animal (puede que sobrepase limite de hambre y no muera)
- tamano lobos y ovejas (la disminucion del hambre dependera del tamano de la oveja y si fue compartida o no)
- lobos grandes necesitan mas alimento que lobos pequenos
Sino se hace, considerarlo como trabajo a futuro

¿Qu\'e hace mi co\'odigo hasta ahora?
- Se crea una poblaci\'on inicial con cierta cantidad de individuos
- De manera aleatoria se decide quien come y quien no (version 1.0 solo lobos comen)
- Para los lobos que van a comer, se revisa la distancia con todas las ovejas, las que queden dentro de un radio seran comidad
- Actualizo poblacion
- Reproduzco las ovejas
- Reproduzco y mato lobos
- movimiento (desplazamiento)

%--------------------------------------------

21 nov

- agregar interaccion pasto
- reporte version 1.0
- pruebas version 1.0 (modificar probabilidades, observar efectos)

